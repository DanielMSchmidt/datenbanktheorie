\documentclass[12pt,a4paper]{amsart}

\synctex = 1

\DeclareMathAlphabet{\mathpzc}{OT1}{pzc}{m}{it}

\def\Bb{\mathbb{B}}
\def\C{\mathbb{C}}
\def\E{\mathbb{E}}
\def\F{\mathbb{F}}
\def\Hh{\mathbb{H}}
\def\K{\mathbb{K}}
\def\N{\mathbb{N}}
\def\P{\mathbb{P}}
\def\Q{\mathbb{Q}}
\def\R{\mathbb{R}}
\def\Z{\mathbb{Z}}



%A
\DeclareMathOperator{\akte}{akte}
%B
\DeclareMathOperator{\bearbeitet}{bearbeitet}
\DeclareMathOperator{\Black}{Black}
%C
%D
\DeclareMathOperator{\deff}{def}
\DeclareMathOperator{\Dom}{Dom}
%E
\DeclareMathOperator{\Einbruch}{Einbruch}
\DeclareMathOperator{\ermittler}{ermittler}
\DeclareMathOperator{\ext}{ext}
%F
%G
\DeclareMathOperator{\gdw}{gdw}
%H
%I
%J
%K
%L
\DeclareMathOperator{\wle}{le}
\DeclareMathOperator{\ltdermittler}{ltdermittler}
%M
\DeclareMathOperator{\Mulder}{Mulder}
%N
\DeclareMathOperator{\Name}{Name}
%O
\DeclareMathOperator{\offen}{offen}
%P
\DeclareMathOperator{\Pathologie}{Pathologie}
%Q
%R
%S
\DeclareMathOperator{\Skinner}{Skinner}
\DeclareMathOperator{\Skully}{Skully}
\DeclareMathOperator{\Sonderermittler}{Sonderermittler}
%T
%U
%V
%X
%Z

\def\fspace{\rule{0.5cm}{0cm}}
\theoremstyle{definition}
\newtheorem{aufgabe1}{Aufgabe}

\usepackage{amsmath,amsthm,amssymb}
\usepackage[utf8]{inputenc}
\usepackage{wasysym}
\usepackage{stmaryrd}
\usepackage{nicefrac}
\usepackage{listings}
\usepackage{pictex}
\usepackage{color}
\usepackage{graphicx}
\usepackage{german}

%%%%%%%%%%%%%%%%%%%%%%%%%%%%%%%%%%%%%%%%%%%%%%%%%%%%%%%%%%%%%%%%%%%%%%%

\begin{document}

\title{Blatt 3}

\author{Pamela Fleischmann \& Daniel Schmidt}
\maketitle


\begin{aufgabe1}
Diskutieren Sie analog zur Vorlesung die Klauselformen des Typs 5 und 6.
\end{aufgabe1}

\medskip

Für $n\geq 1$ ist eine Klausel von Typ 5 gegeben durch
\[
q_1(\dots),\dots,q_n(\dots):-.
\]
Da der Kopf der Formel als Disjunktion interpretiert wird, entspricht diese Klauselform unvollständigem oder ungenauem Wissen. Mindestens eines
der Prädikate muss wahr sein, es ist aber nicht klar welches. In der Theorie tritt ein Problem auf, da mit dem Vollständigkeitsaxiom auch $\lnot q_1(\dots),\dots,
\lnot q_n(\dots)$ ableitbar sind. Zusammen mit der Klausel ergäbe dies eine inkonsistente Theorie. Um diese Inkonsistenz zu beseitigen muss garantiert werden,
dass spezifische Negate nicht mehr ableitbar sind, wenn diese Regel vorhanden ist. Dies kann darüber erreicht werden, dass nur noch bestimmte Belegungen
für die Variablen per Regel zugelassen werden. Weiter ist anzumerken, dass mit dieser Art der Klausel auch Fakten abgeleitet werden können: Ist $j\in[n]$
und sind $\lnot q_i(\dots)$ für $i\in[n]\backslash\{j\}$ ableitbar, so muss wegen dieser Regel $q_j(\dots)$ ein gültiges Faktum sein.

\medskip

Bei Typ 6 kann das ungenaue Wissen nun noch von weiteren Fakten abhängen. Da analoge Inkonsistenzen in der Theorie auftreten können, muss durch
eine geeignte Anpassung wieder sichergestellt werden, dass Negate nicht aufgenommen werden können. Auch dieser Typ kann als Integritätsbedingung benutzt werden, wenn
$q_i$ als das Gleichheitsprädikat angenommen wird.

\bigskip

\begin{aufgabe1}
Beispiele für Typ 5 und 6 anhand von Aufgabe 2 aus Serie 2.
\end{aufgabe1}

\medskip
Typ 5: Irgendwer muss in der Verwaltung arbeiten. \\
\begin{equation}
\begin{split}
\ermittler(\Black,\Verwaltung)&, \\
\ermittler(\Mulder,\Verwaltung)&, \\
\ermittler(\Skinner,\Verwaltung)&, \\
\ermittler(\Skully,\Verwaltung)&:-.
\end{split}
\end{equation}

Typ 6: Jeder offene Fall wird von Mulder, Skully oder beiden bearbeitet. \\
$\bearbeitet(\Mulder,X),\bearbeitet(\Skully,X):-\akte(X,\offen,Y)$.


\bigskip

\begin{aufgabe1}
\end{aufgabe1}

ad a) \\

\paragraph{DRC}
$() / \lnot (\exists X)(\exists Y)(\exists Z) ((AKTE(X, \offen, Y) \wedge AKTE(X, \geloest, Z)) \Rightarrow Y = Z)$ \\
\paragraph{Klauselform}
$(X, Y) :- \akte(F, X, E), \akte(F, Y, E)$ \\
\paragraph{Typ}
6 \\

\bigskip

ad b) \\

\paragraph{DRC}
$(\exists X)(\exists Y) (LTD\_ERMITTLER(X, Y))$ \\
\paragraph{Klauselform}
$lt\_ermittler(X, Y) :- .$\\
\paragraph{Typ}
4 \\

\bigskip

ad c) \\

\paragraph{DRC}
$(\exists X)(BEARBEITET(\Mulder, X) \Rightarrow BEARBEITET(\Skully, X)) \wedge (\exists Y)(BEARBEITET(\Skully, Y) \Rightarrow BEARBEITET(\Mulder, Y))$ \\
\paragraph{Klauselform}
$\team(X,Y) :- \bearbeitet(X, F), \bearbeitet(Y, F).$ \\
$\team(\Mulder, \Skully) :-.$

\paragraph{Typ}
4 und 6 \\
\bigskip


ad d) \\

\paragraph{DRC}
$(\forall E)(AKTE(\Krycek, \offen, E) \Rightarrow BEARBEITET(\Mulder, \Krycek))$ \\
\paragraph{Klauselform}
$bearbeitet(\Mulder, \Krycek) :- \akte(\Krycek, \offen, X)$\\
\paragraph{Typ}
6 \\
\bigskip


ad e) \\

\paragraph{DRC}

$(\forall X)(\forall Y)(\forall Z)(ERMITTLER(X, Y) \\$
$\wedge (Y \neq \Sonderermittler \wedge Y \neq \Verwaltung)) \Rightarrow \bearbeitet(X,Z))$

\paragraph{Klauselform}
$ermittler2 :- \ermittler(X,Y), Y \neq \Sonderermittler, y \neq \Verwaltung.$
\paragraph{Typ}
4 \\

\bigskip

ad f) \\

\paragraph{DRC}
$\lnot (\forall X)(\forall Y)(\forall Z)(AKTE(X, \geloest, Y) \Rightarrow \lnot \bearbeitet(X, Z))$

\paragraph{Klauselform}
$:- \lnot \bearbeitet(X, Z), \akte(X, \geloest, Y).$
\paragraph{Typ}
3 \\
\end{document}
