\documentclass[12pt,a4paper]{amsart}

\synctex = 1

\DeclareMathAlphabet{\mathpzc}{OT1}{pzc}{m}{it}

\def\Bb{\mathbb{B}}
\def\C{\mathbb{C}}
\def\E{\mathbb{E}}
\def\F{\mathbb{F}}
\def\Hh{\mathbb{H}}
\def\K{\mathbb{K}}
\def\N{\mathbb{N}}
\def\P{\mathbb{P}}
\def\Q{\mathbb{Q}}
\def\R{\mathbb{R}}
\def\Z{\mathbb{Z}}



%A
\DeclareMathOperator{\akte}{akte}
%B
\DeclareMathOperator{\bearbeitet}{bearbeitet}
\DeclareMathOperator{\Black}{Black}
%C
\DeclareMathOperator{\cons}{cons}
%D
\DeclareMathOperator{\deff}{def}
\DeclareMathOperator{\Dom}{Dom}
%E
\DeclareMathOperator{\Einbruch}{Einbruch}
\DeclareMathOperator{\ermittler}{ermittler}
\DeclareMathOperator{\ext}{ext}
%F
%G
\DeclareMathOperator{\gdw}{gdw}
%H
\DeclareMathOperator{\HB}{HB}
%I
%J
%K
\DeclareMathOperator{\Konst}{Konst}
%L
\DeclareMathOperator{\wle}{le}
\DeclareMathOperator{\ltdermittler}{ltdermittler}
%M
\DeclareMathOperator{\Mulder}{Mulder}
%N
\DeclareMathOperator{\Name}{Name}
%O
\DeclareMathOperator{\offen}{offen}
%P
\DeclareMathOperator{\Pathologie}{Pathologie}
\DeclareMathOperator{\Pred}{Pred}
%Q
%R
%S
\DeclareMathOperator{\Skinner}{Skinner}
\DeclareMathOperator{\Skully}{Skully}
\DeclareMathOperator{\Sonderermittler}{Sonderermittler}
%T
%U
%V
\DeclareMathOperator{\Verwaltung}{Verwaltung}
%X
%Z

\def\fspace{\rule{0.5cm}{0cm}}
\theoremstyle{definition}
\newtheorem{aufgabe1}{Aufgabe}

\usepackage{amsmath,amsthm,amssymb}
\usepackage[utf8]{inputenc}
\usepackage{wasysym}
\usepackage{stmaryrd}
\usepackage{nicefrac}
\usepackage{listings}
\usepackage{pictex}
\usepackage{color}
\usepackage{graphicx}
\usepackage{german}

\lstset{basicstyle=\small}

%%%%%%%%%%%%%%%%%%%%%%%%%%%%%%%%%%%%%%%%%%%%%%%%%%%%%%%%%%%%%%%%%%%%%%%

\begin{document}

\title{Blatt 9}

\author{Daniel Schmidt \& Pamela Fleischmann}

\maketitle

\begin{aufgabe1}
Um zu zeigen, dass Datalog ohne Rekursion, aber mit Negation und sicheren Regeln die gleiche Ausdruckskraft wie die Relationenalgebra hat muss gezeigt werden, dass $\mu(L) = \mu'(L')$ gilt, wobei Datalog als $(L, \mu)$ definiert ist und die Relationenalgebra $(L', \mu')$ ist.
Hierzu gilt zu zeigen, dass für ein beliebiges Datenbankschema $\sigma$ gilt



\end{aufgabe1}


\begin{aufgabe1}
ad. a)

Die äquivalente Darstellung in Prolog ist

\begin{lstlisting}
a(A,B) :- R(A,B,_).
a(A,B) :- S(A,D), T(_,D,B).
\end{lstlisting}

ad. b)

Die äquivalente Darstellung in Prolog ist


\begin{lstlisting}
sbt(A,B,C,G) :- S(A,C), T(B,C,G).
r'(A,B,C) :- R(A,B,C), A > B.
r'(A,B,C) :- R(A,B,C), C > 5.
rbs(A,B,C,G) :- r'(A,B,C), S(A,G).
q(A,B,C,G) :- sbt(A,B,C,G), not rbs(A,B,C,G).
\end{lstlisting}
\end{aufgabe1}

\begin{aufgabe1}
\end{aufgabe1}

\end{document}
