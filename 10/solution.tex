\documentclass[12pt,a4paper]{amsart}

\synctex = 1

\DeclareMathAlphabet{\mathpzc}{OT1}{pzc}{m}{it}

\def\Bb{\mathbb{B}}
\def\C{\mathbb{C}}
\def\E{\mathbb{E}}
\def\F{\mathbb{F}}
\def\Hh{\mathbb{H}}
\def\K{\mathbb{K}}
\def\N{\mathbb{N}}
\def\P{\mathbb{P}}
\def\Q{\mathbb{Q}}
\def\R{\mathbb{R}}
\def\Z{\mathbb{Z}}



%A
\DeclareMathOperator{\akte}{akte}
%B
\DeclareMathOperator{\bearbeitet}{bearbeitet}
\DeclareMathOperator{\Black}{Black}
%C
%D
\DeclareMathOperator{\deff}{def}
\DeclareMathOperator{\Dom}{Dom}
%E
\DeclareMathOperator{\Einbruch}{Einbruch}
\DeclareMathOperator{\ermittler}{ermittler}
\DeclareMathOperator{\ext}{ext}
%F
%G
\DeclareMathOperator{\gdw}{gdw}
%H
%I
%J
%K
%L
\DeclareMathOperator{\wle}{le}
\DeclareMathOperator{\ltdermittler}{ltdermittler}
%M
\DeclareMathOperator{\Mulder}{Mulder}
%N
\DeclareMathOperator{\Name}{Name}
%O
\DeclareMathOperator{\offen}{offen}
%P
\DeclareMathOperator{\Pathologie}{Pathologie}
%Q
%R
%S
\DeclareMathOperator{\Skinner}{Skinner}
\DeclareMathOperator{\Skully}{Skully}
\DeclareMathOperator{\Sonderermittler}{Sonderermittler}
%T
%U
%V
%X
%Z

\def\fspace{\rule{0.5cm}{0cm}}
\theoremstyle{definition}
\newtheorem{aufgabe1}{Aufgabe}

\DeclareMathOperator{\dom}{dom}
\DeclareMathOperator{\vga}{vga}

\usepackage{amsmath,amsthm,amssymb}
\usepackage[utf8]{inputenc}
\usepackage{wasysym}
\usepackage{stmaryrd}
\usepackage{nicefrac}
\usepackage{listings}
\usepackage{pictex}
\usepackage{color}
\usepackage{graphicx}
\usepackage{german}

\lstset{basicstyle=\small}

%%%%%%%%%%%%%%%%%%%%%%%%%%%%%%%%%%%%%%%%%%%%%%%%%%%%%%%%%%%%%%%%%%%%%%%

\begin{document}

\title{Blatt 10}

\author{Daniel Schmidt \& Pamela Fleischmann}

\maketitle

\begin{aufgabe1}
Um zu zeigen, dass sich für jede TRC-Anfrage zu einem DB-Schema $\sigma$ eine äquivalente Anfrage des DRC zu $\sigma$ finden lässt definieren wir uns einen Algorithmus, welcher TRC-Anfragen zu DRC-Anfragen umformt.
Sei also eine allgemeine TRC-Anfrage $(x) / \theta(x)$, so lässt sich der Algorithmus wie folgt beschreiben: \\
Sei zunächst für jede Variable $k$ in $x$ mit dem Typen $\tau_1, \cdots, \tau_n$ neue Variablen $k_1, \cdots, k_n$ mit den entsprechenden Typen eingeführt. Nun gilt es die Variablen zu ersetzen um DRC-Anfragen zu erhalten, dies geschieht nach den folgenden Regeln: \\
Wenn $RT_i(k)$ gegeben ist, so muss dies durch $RT_i(k_1, \cdots, k_n)$ ersetzt werden.
Falls $k.B_j \theta c(c \theta k.B_j)$ gegeben ist, so muss dies durch $k_j \theta c(c \theta k_j)$ ersetzt werden.
Wenn $k.B_j \theta z.C_h$ gegeben ist, so muss dies ersetzt werden durch $k_j \theta z_h$. \\
Falls $\exists k$ gegeben ist, so muss dies falls $k$ gebunden ist durch $(\exists k_1), \cdots, (\exists k_n)$ ersetzt werden. Falls $k$ ungebunden ist, so ist dies nicht nötig, da das Ergbnis ohnehin nicht weiterverwendet wird.
Analog lässt sich $\forall k$ umformen.
Zuguterletzt muss die Zielfunktion noch angepasst werden, entprechend also $(x) / \cdots$ zu $(x_1, \cdots, x_n) / \cdots$ umgeformt werden.
\end{aufgabe1}

\begin{aufgabe1}
\end{aufgabe1}

\end{document}
