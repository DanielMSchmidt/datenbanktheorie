\documentclass[12pt,a4paper]{amsart}

\synctex = 1

\DeclareMathAlphabet{\mathpzc}{OT1}{pzc}{m}{it}

\def\Bb{\mathbb{B}}
\def\C{\mathbb{C}}
\def\E{\mathbb{E}}
\def\F{\mathbb{F}}
\def\Hh{\mathbb{H}}
\def\K{\mathbb{K}}
\def\N{\mathbb{N}}
\def\P{\mathbb{P}}
\def\Q{\mathbb{Q}}
\def\R{\mathbb{R}}
\def\Z{\mathbb{Z}}



%A
\DeclareMathOperator{\akte}{akte}
%B
\DeclareMathOperator{\bearbeitet}{bearbeitet}
\DeclareMathOperator{\Black}{Black}
%C
%D
\DeclareMathOperator{\deff}{def}
\DeclareMathOperator{\Dom}{Dom}
%E
\DeclareMathOperator{\Einbruch}{Einbruch}
\DeclareMathOperator{\ermittler}{ermittler}
\DeclareMathOperator{\ext}{ext}
%F
%G
\DeclareMathOperator{\gdw}{gdw}
%H
%I
%J
%K
%L
\DeclareMathOperator{\wle}{le}
\DeclareMathOperator{\ltdermittler}{ltdermittler}
%M
\DeclareMathOperator{\Mulder}{Mulder}
%N
\DeclareMathOperator{\Name}{Name}
%O
\DeclareMathOperator{\offen}{offen}
%P
\DeclareMathOperator{\Pathologie}{Pathologie}
%Q
%R
%S
\DeclareMathOperator{\Skinner}{Skinner}
\DeclareMathOperator{\Skully}{Skully}
\DeclareMathOperator{\Sonderermittler}{Sonderermittler}
%T
%U
%V
%X
%Z

\def\fspace{\rule{0.5cm}{0cm}}
\theoremstyle{definition}
\newtheorem{aufgabe1}{Aufgabe}

\usepackage{amsmath,amsthm,amssymb}
\usepackage[utf8]{inputenc}
\usepackage{wasysym}
\usepackage{stmaryrd}
\usepackage{nicefrac}
\usepackage{listings}
\usepackage{pictex}
\usepackage{color}
\usepackage{graphicx}
\usepackage{german}


\lstset{
    frame=single,
    breaklines=true,
}

%%%%%%%%%%%%%%%%%%%%%%%%%%%%%%%%%%%%%%%%%%%%%%%%%%%%%%%%%%%%%%%%%%%%%%%

\begin{document}

\title{Blatt 4}

\author{Daniel Schmidt \& Pamela Fleischmann}

\maketitle

\begin{aufgabe1}
Sei $V=(\mathcal{P}(\N),\subseteq)$ und $R\subseteq\N$. Dann gilt\\
a. $\tau(M)=\{x-1|\,x\in M\backslash\{1\}\}$ ist monoton und die leere Menge ist der einzige Fixpunkt.\\
b. $\tau(M)=M\backslash R$ ist monoton und genau die $F\in\mathcal{P}(\N)$ mit $F\cap R=\emptyset$ sind die Fixpunkte.\\
c. $\tau(M)=R\backslash M$ ist nicht monoton und hat keine Fixpunkte.\\
d. $\tau(M)=\{1\}\cup\{2x|\,x\in M\}$ ist monoton und $F=\{1\}\cup\{n\in\N|\,n\equiv_20\}$ ist der einzige Fixpunkt.
\end{aufgabe1}

\medskip

{\em Beweis.}
ad a. Seien $M_1,M_2\in\mathcal{P}(\N)$ mit $M_1\subseteq M_2$. Seit weiter $m\in\tau(M_1)$. Dann existiert ein $n\in M_1\backslash\{1\}$ mit
$m=n-1$. Wegen $M_1\subseteq M_2$ gilt $n\in M_2\backslash\{1\}$. Damit gilt $m\in\tau(M_2)$ und $\tau$ ist monoton.\\
Für alle $x\in\emptyset$ gilt offensichtlich $x-1\in\emptyset$ und somit ist die leere Menge ein Fixpunkt von $\tau$.\\
Annahme: $\exists M\in\mathcal{P}\backslash\{\emptyset\}:\,\tau(M)=M$\\
Wegen $M\neq\emptyset$ gilt $|M|\geq 1$. Ist $M=\{1\}$, so ist $\tau(M)=\emptyset\neq M$. $\lightning$ Somit existiert in $M$ mindestens ein $m\neq 1$. Wähle
dieses maximal. Wegen $M=\tau(M)$ gilt $m\in\tau(M)$ und somit existiert ein $n\in M\backslash\{1\}$ mit $m=n-1$. Damit gilt $m+1=n\in M$, was ein Widerspruch zur
Maximalität von $m$ ist. Damit ist $\emptyset$ der einzige Fixpunkt von $\tau$.

\medskip

ad b. Seien $M_1,M_2\in\mathcal{P}(\N)$ mit $M_1\subseteq M_2$. Seit weiter $m\in\tau(M_1)$. Dann gilt $x\in M_1\backslash R$, also $x\in M_1$ und $x\not\in R$.
Wegen $M_1\subseteq M_2$ gilt $x\in M_2$ und es folgt $x\in M_2\backslash R=\tau(M_2)$. Damit ist $\tau$ monoton.\\
Für $F\in\mathcal{P}(\N)$ mit $F\cap R=\emptyset$ gilt offensichtlich $\tau(M)=F\backslash R =F$. \\
Annahme: $\exists M\in\mathcal{P}(\N):\,M\cap R\neq\emptyset\land \tau_(M)=M$\\
Dann gilt $\tau(M)=M\backslash R\subset M$. $\lightning$ Damit sind die zu $R$ disjunkten Teilmengen die einzigen Fixpunkte von $\tau$.

\medskip

ad c. Annahme: $\tau$ ist monoton.\\
Seien $M_1,M_2\in\mathcal{P}(\N)$ mit $M_1\subset M_2$ und $\tau(M_1)\subseteq \tau(M_2)$. Sei $x\in R\cap M_2\backslash M_1$. Dann gilt $x\not\in M_1$ und $x\in R$
und somit $x\in R\backslash M_1=\tau(M_1)$. Damit gilt $x\in \tau(M_2)=R\backslash M_2$, also auch $x\not\in M_2$. $\lightning$.\\
Annahme: $\exists M\in\mathcal{P}(\N)$ mit $\tau(M)=M$\\
Dann gilt für jedes $m\in M$ aber $m\in\tau(M)=R\backslash M$ also $m\not\in M$. $\lightning$

ad d. Seien $M_1,M_2\in\mathcal{P}(\N)$ mit $M_1\subseteq M_2$. Seit weiter $m\in\tau(M_1)$. Ist $m=1$, so ist $m$ offensichtlich Element von $\tau(M_2)$. Sei also
$m=2n$ für ein passendes $n\in M_1$. Wegen $M_1\subseteq M_2$ gilt $n\in M_2$ und somit $m\in\tau(M_2)$.\\
Ist $x\in F$, so gilt entweder $x=1$ oder $x=2\ell$ für ein $\ell\in\N$. Damit gilt $x\in \tau_(F)$. Ist $x\in\tau(F)$, so ist $x$ entweder $1$ oder es existiert ein
$y\in F$ mit $x=2y$. Im ersten Fall ist $x$ offensichtlich in $\tau(F)$ und im zweiten Fall ist $x$ offensichtlich gerade und somit auch in $\tau(F)$.\\
Annahme: $\exists M\in\mathcal{P}(\N)\backslash \{F\}:\,\tau(M)=M$\\
Wegen $1\in\tau(M)=M$ ist $M$ nicht leer. Wähle $m\in M$ maximal. $m$ kann o.B.d.A. als echt größer $1$ angenommen werden, da mit $1\in M$ schon $2\in\tau(M)=M$ gilt.
Gäbe es ein $\ell$ mit $m=2\ell+1$, so wäre $\ell\geq 1$ und somit $2\ell+1\in\tau(M)$. Damit gälte $2\ell+1=2n$ für ein $n\in M$. $\lightning$ Somit kann $m$ als gerade
vorausgesetzt werden. Damit enthält $M$ außer der $1$ nur gerade Zahlen. Wegen $m\in M$ gilt aber $m<2m\in\tau(M)=M$, was ein Widerspruch zur Maximalität von $m$ ist.\qed

\begin{aufgabe1}
Sei $P$ ein Datalog-Programm. Dann gilt\\
1) Mit je zwei Herbrand-Modellen $I_1$ und $I_2$ von $P$ ist auch $I_1\cap I_2$ ein Herbrand-Modell von $P$.\\
2) Es existiert ein $P$ und es existieren Herbrand-Modelle $I_1$ und $I_2$ von $P$, so dass $I_1\cup I_2$ kein Herbrand-Modell von $P$ ist.\\
3) Die Herbrand-Basis $\HB$ ist ein Herbrand-Modell von $P$.\\
4) Es existiert ein $P$, so dass $\emptyset$ ein Herbrand-Modell von $P$ ist.\\
5) Seien $\Konst_P$ die Menge der in $P$ vorkommenden Konstantensymbole und $I$ ein beliebiges Herbrand-Modell von $P$. Dann ist auch $I= \{p(k_1,\dots, k_j) \in I |\, p\in\Pred, k_1,\dots, k_j \in \Konst_P \}$ ein Herbrand-Modell von P.
\end{aufgabe1}

{\em Beweis.}
ad 1) Seien $I_1$ und $I_2$ Herbrand-Modelle von $P$ und $d\in P$. Ist $d$ ein Grundatom, so gilt $d\in I_1,I_2$ und somit $d\in I_1\cap I_2$. Sei $d$ nun $q(\dots):-
p_1(\dots),\dots,p_m(\dots)$ und $\varrho$ eine Belegung von $d$. Ist ein $||p_i(\dots)||_{\varrho}$ ($i\in[m]$) weder in $I_1$ noch in $I_2$, so gilt auch 
$||p_i(\dots)||_{\varrho}\not\in I_1\cap I_2$ und wegen der falschen Prämisse ist die Regel gültig und es gilt $\models_{I_1\cap I_2,\varrho}d$. Sind für alle $i\in[m]$ $p_i(\dots)\in I_1,I_2$, so gilt die Behauptung offensichtlich. Seien nun $||p_{i_1}||_{\varrho},\dots,||p_{i_k}||_{\varrho}\in I_1$ und $||p_{j_1}||_{\varrho},\dots,||p_{j_{\ell}}||_{\varrho}\in I_2$ mit $i_1,\dots,i_k,j_1,\dots,j_{\ell}\in[m]$ und $\{i_1,\dots,i_k\}\cap\{j_1\dots,j_k\}\neq[m],\emptyset$. Dann existiert ein $c\in[m]$,
so dass $||p_c(\dots)||_{\varrho}\not\in I_1\cap I_2$ gilt. Damit ist die Prämisse falsch und die Regel gültig unter $I_1\cap I_2$.

\medskip

ad 2) Betrachte $P=\{p(X):-q(X),r(X)\}$ sowie $I_1=\{q(1)\}$ und $I_2=\{r(1)\}$ für $\Konst_P=\{1\}$. Dann sind $I_1$ und $I_2$ Herbrand-Modelle, da in beiden Fällen
die Prämissen falsch sind. $I_1\cup I_2=\{r(1),q(1)\}$ ist allerdings kein Herbrand-Modell für $P$, da die Prämisse wahr ist, $q(1)$ aber nicht in $I_1\cup I_2$ enthalten ist.

\medskip

ad 3) In $\HB_P$ sind alle Grundatome enthalten. Da $\cons(P)$ ein Herbrand-Modell ist, $\cons(P)\subseteq\HB_P$ gilt und $\HB_P$ maximal ist, ist $\HB_P$ ein Herbrand-Modell von $P$.

\medskip

ad 4) Betrachte $P=\{:-p(X)\}$. Da kein Element in der leeren Menge ist, ist die Prämisse falsch und Regel somit gültig.

\medskip

ad 5) Da $I$ ein Modell ist, sind die Grundatome in $I$ und somit auch in $I'$. Da für alle Konstanten genau die Klauseln aufgenommen werden, die auch in $I$ sind und
$I$ ein Herbrand-Modell ist, überträgt sich die Eigenschaft auf $I'$.

% TODO: Bei silbling fehlt generell noch ungleichheit
\begin{aufgabe1}
Geben Sie eine natürlichsprachige Definition folgender Verwandschaftsbeziehungen an, übersetzen Sie diese in reines Datalog und testen Sie Ihre Lösung mit CORAL:\\
a) elternteil
b) großvater
c) schwester (incl. Halbschwester)
d) onkel
e) nichte
f) vorfahre
g) sind\_verwand
\end{aufgabe1}

ad a)

Die natürlichsprachige Definition wäre ``Vater oder Mutter'', das Datalog Programm sieht entsprechend so aus:

\begin{lstlisting}
module serie.
export parent(ff,fb,bf).

parent(X,Y) :- vater(X,Y).
parent(X,Y) :- mutter(X,Y).

end_module.
\end{lstlisting}

\medskip

ad b)

Die natürlichsprachige Definition wäre ``Vater oder Muter von Vater oder Mutter'', das Datalog Programm sieht dementsprechend so aus:

\begin{lstlisting}
module serie.
export parent(ff,fb,bf), grandparent(ff,bf,fb).

parent(X,Y) :- vater(X,Y).
parent(X,Y) :- mutter(X,Y).

grandparent(X,Z) :- parent(X,Y), parent(Y,Z).

end_module.
\end{lstlisting}

\medskip

ad c)

Die natürlichsprachige Definition wäre ``Vater oder Muter von beiden Eingaben sind identisch und die zweite Eingabe (Schwester) ist weiblich'', das Datalog Programm sieht so aus:

\begin{lstlisting}
module serie.
export parent(ff,fb,bf), grandparent(ff,bf,fb), sister(ff,fb,bf).

parent(X,Y) :- vater(X,Y).
parent(X,Y) :- mutter(X,Y).

grandparent(X,Z) :- parent(X,Y), parent(Y,Z).
sister(X,Y) :- parent(Z, X), parent(Z, Y), weiblich(Y).

end_module.
\end{lstlisting}

\medskip

ad d)

Die natürlichsprachige Definition wäre ``Bruder oder Schwester meines Vaters oder meiner Mutter'', das Datalog Programm sieht so aus:

\begin{lstlisting}
module serie.
export parent(ff,fb,bf), grandparent(ff,bf,fb), sister(ff,fb,bf), uncle(ff,fb,bf).

parent(X,Y) :- vater(X,Y).
parent(X,Y) :- mutter(X,Y).

grandparent(X,Z) :- parent(X,Y), parent(Y,Z).
sister(X,Y) :- parent(Z, X), parent(Z, Y), weiblich(Y).
silbling(X,Y) :- parent(Z,X), parent(Z, Y).
uncle(X,Y) :- parent(Z,Y), silbling(Z,X), maennlich(X).

end_module.
\end{lstlisting}

\medskip

ad e)

Die natürlichsprachige Definition wäre ``Tochter meiner Schwester oder meines Bruders'', das Datalog Programm sieht so aus:

\begin{lstlisting}
module serie.
export parent(ff,fb,bf), grandparent(ff,bf,fb), sister(ff,fb,bf), uncle(ff,fb,bf), niece(ff,fb,bf).

parent(X,Y) :- vater(X,Y).
parent(X,Y) :- mutter(X,Y).

grandparent(X,Z) :- parent(X,Y), parent(Y,Z).
sister(X,Y) :- parent(Z, X), parent(Z, Y), weiblich(Y).
silbling(X,Y) :- parent(Z,X), parent(Z, Y).
uncle(X,Y) :- parent(Z,Y), silbling(Z,X), maennlich(X).
niece(X,Y) :- parent(Z,X), silbling(Y,Z), weiblich(X).

end_module.
\end{lstlisting}

\medskip

ad f)
Die natürlichsprachige Definition wäre ``Ein einer vorherigen Generation angehöriger Verwandter'', das Datalog Programm sieht so aus:

\begin{lstlisting}
module serie.
export parent(ff,fb,bf), grandparent(ff,bf,fb), sister(ff,fb,bf), uncle(ff,fb,bf), niece(ff,fb,bf), ancestor(ff,fb,bf).

parent(X,Y) :- vater(X,Y).
parent(X,Y) :- mutter(X,Y).

grandparent(X,Z) :- parent(X,Y), parent(Y,Z).
sister(X,Y) :- parent(Z, X), parent(Z, Y), weiblich(Y).
silbling(X,Y) :- parent(Z,X), parent(Z, Y).
uncle(X,Y) :- parent(Z,Y), silbling(Z,X), maennlich(X).
niece(X,Y) :- parent(Z,X), silbling(Y,Z), weiblich(X).
ancestor(X,Y) :- parent(X, Y).
ancestor(X,Y) :- parent(X, Z), ancestor(Z,Y).

end_module.
\end{lstlisting}

\medskip

ad g)

Die natürlichsprachige Definition ``es gibt einen gemeinsamen Vorfahren'' ist schon in der Aufgabenstellung gegeben, das Datalog Programm sähe entsprechend so aus:

\begin{lstlisting}
module serie.
export parent(ff,fb,bf), grandparent(ff,bf,fb), sister(ff,fb,bf), uncle(ff,fb,bf), niece(ff,fb,bf), ancestor(ff,fb,bf), is_related(ff,fb,bf).

parent(X,Y) :- vater(X,Y).
parent(X,Y) :- mutter(X,Y).

grandparent(X,Z) :- parent(X,Y), parent(Y,Z).
sister(X,Y) :- parent(Z, X), parent(Z, Y), weiblich(Y).
silbling(X,Y) :- parent(Z,X), parent(Z, Y).
uncle(X,Y) :- parent(Z,Y), silbling(Z,X), maennlich(X).
niece(X,Y) :- parent(Z,X), silbling(Y,Z), weiblich(X).
ancestor(X,Y) :- parent(X, Y).
ancestor(X,Y) :- parent(X, Z), ancestor(Z,Y).

is_related(X,Y) :- ancestor(X,Z), ancestor(Y,Z), not equal(X,Y).

end_module.
\end{lstlisting}

Dementsprechend kommt dieses Beispiel nicht ohne build-in prädikate aus.

\medskip
\end{document}
