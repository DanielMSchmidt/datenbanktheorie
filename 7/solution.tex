\documentclass[12pt,a4paper]{amsart}

\synctex = 1

\DeclareMathAlphabet{\mathpzc}{OT1}{pzc}{m}{it}

\def\Bb{\mathbb{B}}
\def\C{\mathbb{C}}
\def\E{\mathbb{E}}
\def\F{\mathbb{F}}
\def\Hh{\mathbb{H}}
\def\K{\mathbb{K}}
\def\N{\mathbb{N}}
\def\P{\mathbb{P}}
\def\Q{\mathbb{Q}}
\def\R{\mathbb{R}}
\def\Z{\mathbb{Z}}



%A
\DeclareMathOperator{\akte}{akte}
%B
\DeclareMathOperator{\bearbeitet}{bearbeitet}
\DeclareMathOperator{\Black}{Black}
%C
%D
\DeclareMathOperator{\deff}{def}
\DeclareMathOperator{\Dom}{Dom}
%E
\DeclareMathOperator{\Einbruch}{Einbruch}
\DeclareMathOperator{\ermittler}{ermittler}
\DeclareMathOperator{\ext}{ext}
%F
%G
\DeclareMathOperator{\gdw}{gdw}
%H
%I
%J
%K
%L
\DeclareMathOperator{\wle}{le}
\DeclareMathOperator{\ltdermittler}{ltdermittler}
%M
\DeclareMathOperator{\Mulder}{Mulder}
%N
\DeclareMathOperator{\Name}{Name}
%O
\DeclareMathOperator{\offen}{offen}
%P
\DeclareMathOperator{\Pathologie}{Pathologie}
%Q
%R
%S
\DeclareMathOperator{\Skinner}{Skinner}
\DeclareMathOperator{\Skully}{Skully}
\DeclareMathOperator{\Sonderermittler}{Sonderermittler}
%T
%U
%V
%X
%Z

\def\fspace{\rule{0.5cm}{0cm}}
\theoremstyle{definition}
\newtheorem{aufgabe1}{Aufgabe}

\usepackage{amsmath,amsthm,amssymb}
\usepackage[utf8]{inputenc}
\usepackage{wasysym}
\usepackage{stmaryrd}
\usepackage{nicefrac}
\usepackage{listings}
\usepackage{pictex}
\usepackage{color}
\usepackage{graphicx}
\usepackage{german}

\lstset{basicstyle=\small}

%%%%%%%%%%%%%%%%%%%%%%%%%%%%%%%%%%%%%%%%%%%%%%%%%%%%%%%%%%%%%%%%%%%%%%%

\begin{document}

\title{Blatt 7}

\author{Daniel Schmidt \& Pamela Fleischmann}

\maketitle

\begin{aufgabe1}
Sei das folgende Datalog-Programm gegeben:

\begin{align*}
&gn(X,Y) :- gl(X,Y). \\
&gn(X,Y) :- kp(X,X1), gn(Y1,X1), kp(Y,Y1).
\end{align*}

und die Zielklausel $? - gn(c,Y).$

\paragraph{Schritt 1:}
\begin{align*}
&r_0 = query^{f}(Y) :- gn^{bf}(c,Y). \\
&r_1 = gn^{bb}(X,Y) :- gl(X,Y). \\
&r_2 = gn^{bf}(X,Y) :- kp(X,X1), gn^{bb}(Y1,X1), kp(Y,Y1). \\
&r_3 = gn^{fb}(X,Y) :- kp(X,X1), gn^{bb}(Y1,X1), kp(Y,Y1).
\end{align*}

\paragraph{Schritt 2:}

\begin{align*}
&magic\_r_0\_gn^{bf}(c) :- . \\
&magic\_r_2\_gn^{fb}(Y) :- gn^{bb}(Y1,X1), kp(Y,Y1). \\
&magic\_r_3\_gn^{bf}(X) :- kp(X,X1), gn^{bb}(Y1,X1).
\end{align*}

\paragraph{Schritt 3:}

\begin{align*}
&query^{f}(Y) :- magic\_r_0\_gn^{bf}(c), gn^{bf}(c,Y). \\
&gn^{bf}(X,Y) :- gl^{bf}(X,Y). \\
&gn^{bf}(X,Y) :- magic\_r_2\_gn^{fb}(Y), gn^{bf}(Y1,X1), kp(Y,Y1). \\
&gn^{fb}(X,Y) :- magic\_r_3\_gn^{bf}(X), kp(X,X1), gn^{fb}(Y1,X1).
\end{align*}

\paragraph{Schritt 4:}

\begin{align*}
&magic\_gn^{bf}(c) :- magic\_r_0\_gn^{bf}(c). \\
&magic\_gn^{bf}(X) :- magic\_r_3\_gn^{bf}(X). \\
&magic\_gn^{fb}(Y) :- magic\_r_2\_gn^{fb}(Y). 
\end{align*}

\paragraph{Resultat:}
\begin{align*}
&query^{f}(Y) :- magic\_gn^{bf}(c), gn^{bf}(c,Y). \\
&magic\_gn^{bf}(c) :- magic\_r_0\_gn^{bf}(c). \\
&magic\_gn^{bf}(X) :- magic\_r_3\_gn^{bf}(X). \\
&magic\_gn^{fb}(Y) :- magic\_r_2\_gn^{fb}(Y). \\
&gn^{bf}(X,Y) :- gl^{bf}(X,Y). \\
&gn^{bf}(X,Y) :- magic\_gn^{fb}(Y), gn^{bf}(Y1,X1), kp(Y,Y1). \\
&gn^{fb}(X,Y) :- magic\_gn^{bf}(X), kp(X,X1), gn^{fb}(Y1,X1).
\end{align*}


\end{aufgabe1}


\begin{aufgabe1}
a. Betrachte das Datalog-Programm
\[
p(a).\,q(X):-\lnot p(X).
\]
Herbrand-Modelle für dieses Programm sind $\{p(a)\}$, $\{p(a),q(b)\}$ und $\{p(a),q(a),q(b)\}$. $r(x,y)$ kann außerdem immer mit dazugenommen werden, da es in der Regel nicht vorkommt. Die Variable $X$ ist unbeschränkt, da sie nicht-negiert nur im Kopf vorkommt. Desweiteren ist $p$ geschichtet, da es im Abhängigkeitsgraphen keinen Zyklus gibt.

\medskip

b. Betrachte das Datalog-Programm
\[
p(X):-\lnot q(X).\,q(X):-\lnot p(X).
\]
Herbrand-Modelle sind $\{q(a),q(b)\}$, $\{p(a),p(b)\}$, $\{q(a),p(b)\}$ und $\{q(b)$, $p(a)\}$. Analog zu a. kann $r$ in allen Varianten wieder dazugenommen werden. $X$ ist in beiden Regeln unbeschränkt. Das Programm ist nicht geschichtet, da der Abhängigkeitsgraph einen Zyklus mit einer mit $\lnot$ beschrifteten Kante enthält.

\medskip

c. Betrachte das Datalog-Programm
\[
p(X):-\lnot q(X).\,q(X):-p(X).
\]
Herbrand-Modell ist $\{p(a),q(a),p(b),q(b)\}$ und wieder kann $r$ in allen Varianten dazugenommen werden. $X$ ist in Regel 1 unbeschränkt, in Regel 2 dagegen beschränkt, da sie im Kopf und im Rumpf nicht-negiert vorkommt. Analog zu b. ist $P$ nicht geschichtet.

\medskip

d. Betrachte das Datalog-Programm
\[
p(a).\,q(b).\,r(X,Y):-\lnot p(X),\lnot q(Y).
\]
Herbrand-Modelle sind $\{p(a),q(b),r(b,a)\}$, $\{p(a),q(b),p(b)\}$, $\{p(a)$, $q(b)$, $q(a)\}$. Beide Variablen sind unbeschränkt, da sie im Rumpf nur negiert vorkommen. Da es keinen Zyklus im Abhängigkeitsgraphen gibt, ist das Programm geschichtet.

\medskip

e. Betrachte das Datalog-Programm
\[
p(a).\,q(b).\,r(X,Y):-p(X),q(Y),\lnot p(X).
\]
Ein Herbrand Modell ist $\{p(a),q(b)\}$. Da der Rumpf nie erfüllt ist, können alle Varianten von $r$ in das Modell mit aufgenommen werden. Die Variable $X$ ist unbeschränkt, die Variable $Y$ ist beschränkt. Das Programm ist geschichtet, da es keinen Zyklus gibt.
\end{aufgabe1}

\medskip

\begin{aufgabe1}
Betrachte das folgende Datalog-Programm mit Negation
\begin{equation*}
\begin{split}
 p(a,b).&\,p(b,c).\,p(b,a).\,p(c,d).\,p(c.c).\\
 q(X,Y) &:- p(X,Y),\lnot p(Y,X).\\
 q(X,X) &:- -q(X,Y).\\
 r(X,Y)&:-\lnot q(Y,X),p(X,Y),\lnot p(X,X).\\
 r(Y,Y)&:-r(X,Y).\\
 s(X,X) &:- \lnot r(X,Y),p(X,Z),p(W,Y).
 \end{split}
\end{equation*}
Für ein perfektes Modell wird die Schichtung betrachtet und somit zuerst das Prädikat $q$. Mit der ersten Regel kommen $q(c,b)$ und $q(d,c)$ hinzu. Mit der zweiten dann
$q(c,c),q(b,b),q(d,d)$. Da $r$ von $p$ und $q$ abhängt, ist $r$ in der folgenden Schicht und es werden $r(a,b)$ und $r(b,a)$ durch Regel 3, sowie $r(b,b)$ und $r(a,a)$ durch
Regel 4 hinzugenommen. Da $s$ nicht von $r$ abhängt, ist $s$ in derselben Schicht und es kommen die Fakten $s(a,a), s(b,b),s(c,c)$ hinzu. Da in keiner Regel, die weiter unten
steht, Elemente der oberen Schichten verändert werden, ist dies das perfekte Modell.
\end{aufgabe1}

\begin{aufgabe1}
ad a.

Die Ausgaben des Original Programms sind wie folgt:
\lstinputlisting{code/a4Before.log}

Folgene Anpassungen an das Programm waren notwendig:
\lstinputlisting{code/flounder.P}

Die resultierende Ausgabe ist die folgende:
\lstinputlisting{code/a4After.log}

\medskip

ad b.

Das Programm sieht wie folgt aus:
\lstinputlisting{code/zshg.P}

Die folgende Ausgabe wird erzeugt:
\lstinputlisting{code/zshg.log}

Leider ist dies nicht die erwartete Ausgabe, g6 müsste eigentlich False ergeben.

\end{aufgabe1}

\end{document}
