\documentclass[12pt,a4paper]{amsart}

\synctex = 1

\DeclareMathAlphabet{\mathpzc}{OT1}{pzc}{m}{it}

\def\Bb{\mathbb{B}}
\def\C{\mathbb{C}}
\def\E{\mathbb{E}}
\def\F{\mathbb{F}}
\def\Hh{\mathbb{H}}
\def\K{\mathbb{K}}
\def\N{\mathbb{N}}
\def\P{\mathbb{P}}
\def\Q{\mathbb{Q}}
\def\R{\mathbb{R}}
\def\Z{\mathbb{Z}}



%A
\DeclareMathOperator{\akte}{akte}
%B
\DeclareMathOperator{\bearbeitet}{bearbeitet}
\DeclareMathOperator{\Black}{Black}
%C
%D
\DeclareMathOperator{\deff}{def}
\DeclareMathOperator{\Dom}{Dom}
%E
\DeclareMathOperator{\Einbruch}{Einbruch}
\DeclareMathOperator{\ermittler}{ermittler}
\DeclareMathOperator{\ext}{ext}
%F
%G
\DeclareMathOperator{\gdw}{gdw}
%H
%I
%J
%K
%L
\DeclareMathOperator{\wle}{le}
\DeclareMathOperator{\ltdermittler}{ltdermittler}
%M
\DeclareMathOperator{\Mulder}{Mulder}
%N
\DeclareMathOperator{\Name}{Name}
%O
\DeclareMathOperator{\offen}{offen}
%P
\DeclareMathOperator{\Pathologie}{Pathologie}
%Q
%R
%S
\DeclareMathOperator{\Skinner}{Skinner}
\DeclareMathOperator{\Skully}{Skully}
\DeclareMathOperator{\Sonderermittler}{Sonderermittler}
%T
%U
%V
%X
%Z

\def\fspace{\rule{0.5cm}{0cm}}
\theoremstyle{definition}
\newtheorem{aufgabe1}{Aufgabe}

\usepackage{amsmath,amsthm,amssymb}
\usepackage[utf8]{inputenc}
\usepackage{wasysym}
\usepackage{stmaryrd}
\usepackage{nicefrac}
\usepackage{listings}
\usepackage{pictex}
\usepackage{color}
\usepackage{graphicx}
\usepackage{german}

\lstset{basicstyle=\small}

%%%%%%%%%%%%%%%%%%%%%%%%%%%%%%%%%%%%%%%%%%%%%%%%%%%%%%%%%%%%%%%%%%%%%%%

\begin{document}

\title{Blatt 7}

\author{Daniel Schmidt \& Pamela Fleischmann}

\maketitle

\begin{aufgabe1}
Sei das folgende Datalog-Programm gegeben:

\begin{align*}
&gn(X,Y) :- gl(X,Y). \\
&gn(X,Y) :- kp(X,X1), gn(Y1,X1), kp(Y,Y1).
\end{align*}

und die Zielklausel $? - gn(c,Y).$

\paragraph{Schritt 1:}
Es gilt die Regel $query^{f}(Y) :- gn^{bf}(c,Y).$ einzufügen, womit das komplette Programm wie folgt lautet: 
\begin{align*}
&r_0 = query^{f}(Y) :- gn^{bf}(c,Y). \\
&r_1 = gn^{bb}(X,Y) :- gl^{ff}(X,Y). \\
&r_2 = gn^{bb}(X,Y) :- kp^{ff}(X,X1), gn^{ff}(Y1,X1), kp^{ff}(Y,Y1).
\end{align*}

\paragraph{Schritt 2:}
Als nächstes müssen alle Vorkommen von IDB-Prädikaten im Rumpf verändert werden, in diesem Fall ist $r_2$, sodass sich folgende Regeln ergeben: 

\begin{align*}
&magic\_r_0\_gn^{bf} :- . \\
&magic\_r_2\_gn^{bb} :- kp^{ff}(X,X1), &magic\_r_2\_gn^{bb}(X1,Y1), kp^{ff}(Y,Y1).
\end{align*}

\end{aufgabe1}


\begin{aufgabe1}
\end{aufgabe1}

\begin{aufgabe1}
\end{aufgabe1}

\begin{aufgabe1}
ad a.

Die Ausgaben des Original Programms sind wie folgt:
\lstinputlisting{code/a4Before.log}

Folgene Anpassungen an das Programm waren notwendig:
\lstinputlisting{code/flounder.P}

Die resultierende Ausgabe ist die folgende:
\lstinputlisting{code/a4After.log}

ad b.

Das Programm sieht wie folgt aus:
\lstinputlisting{code/zshg.P}

Die folgende Ausgabe wird erzeugt:
\lstinputlisting{code/zshg.log}

\end{aufgabe1}

\end{document}
