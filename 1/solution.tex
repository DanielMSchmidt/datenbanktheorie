\documentclass[12pt, a4paper]{article}
\usepackage{url,graphicx,tabularx,array,geometry}
\usepackage[utf8]{inputenc}
\usepackage{paralist}
\usepackage{latexsym}
\usepackage{fancyhdr}
\usepackage{ textcomp }
\usepackage{ mathrsfs }


\pagestyle{fancy}

\usepackage{amsmath}
\usepackage{amsfonts}
\usepackage{amssymb}



\setlength{\parskip}{1ex} %--skip lines between paragraphs
\setlength{\parindent}{0pt} %--don't indent paragraphs

%-- Commands for header
\newcommand{\bs}{\ensuremath{\backslash}}
\renewcommand{\title}[1]{\textbf{#1}\\}
\renewcommand{\line}{\begin{tabularx}{\textwidth}{X>{\raggedleft}X}\hline\\\end{tabularx}\\[-0.5cm]}
\newcommand{\leftright}[2]{\begin{tabularx}{\textwidth}{X>{\raggedleft}X}#1%
& #2\\\end{tabularx}\\[-0.5cm]}
%\linespread{2} %-- Uncomment for Double Space
\begin{document}
\renewcommand{\headrulewidth}{0pt}
\fancyhf{}
\fancyhead[L]{
\leftright{\textbf{Übungsblatt 1}}{Daniel Schmidt}
\line
\leftright{\textbf{Datenbanktheorie SS 16}}{}}
\fancyfoot[C]{\thepage}

\section*{Aufgabe 1}
Sei $Konst_A$ definiert als $\{ a,b \}$ und $Konst_B$ definiert als $\{ a,b,c \}$.
Sei $A_1 = Var \cup Konst_A \cup A_{Rel1}$ und $A_2 = Var \cup Konst_A \cup A_{Rel2}$ definiert, so ist $I_1$ definiert als

$Dom = Konst_{A}$ \\
$p = \{(a,b),(b,a)\}$

so wäre es eine Interpretation die wahr wäre.\\


Sei $I_2$ definiert als 

$Dom = Konst_{B}$ \\
$p = \{(a,b),(b,a), (a,c)\}$

so wäre es eine Interpretation die falsch wäre.

\section*{Aufgabe 2}

\subsection*{a)}

F1 ist nicht erfüllbar, denn hierzu müsse  $p(a,c)$ genau so gelten wie, dass für alle X $p(a,X) \Longrightarrow X = b$ gilt. Das bedeutet, dass $c = b$ gelten müsse	, was nicht wahr sein kann, da Konstanten nicht gleich sein dürfen.\\

F2 ist erfüllbar, da es beispielsweise ein b geben könnte für das entsprechend $b \neq a$ gilt, also $\neg p(b)$ wahr wäre.

\subsection*{b)}

Wie in a) erwähnt ist diese Formel nicht erfüllbar, es gibt aus keine geeignete Relation. Die Begründung hierzu ist in a) gegeben.\\

Eine Interpretation für b) wäre eine Domäne, welche b beinhaltet, dies aber nicht in p enthalten ist, also $\neg p(b)$ gilt.

\section*{Aufgabe 3}



\end{document}
