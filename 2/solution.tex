\documentclass[12pt,a4paper]{amsart}

\synctex = 1

\DeclareMathAlphabet{\mathpzc}{OT1}{pzc}{m}{it}

\def\Bb{\mathbb{B}}
\def\C{\mathbb{C}}
\def\E{\mathbb{E}}
\def\F{\mathbb{F}}
\def\Hh{\mathbb{H}}
\def\K{\mathbb{K}}
\def\N{\mathbb{N}}
\def\P{\mathbb{P}}
\def\Q{\mathbb{Q}}
\def\R{\mathbb{R}}
\def\Z{\mathbb{Z}}



%A
\DeclareMathOperator{\akte}{akte}
%B
\DeclareMathOperator{\bearbeitet}{bearbeitet}
\DeclareMathOperator{\Black}{Black}
%C
%D
\DeclareMathOperator{\deff}{def}
\DeclareMathOperator{\Dom}{Dom}
%E
\DeclareMathOperator{\Einbruch}{Einbruch}
\DeclareMathOperator{\ermittler}{ermittler}
\DeclareMathOperator{\ext}{ext}
%F
%G
\DeclareMathOperator{\gdw}{gdw}
%H
%I
%J
%K
%L
\DeclareMathOperator{\wle}{le}
\DeclareMathOperator{\ltdermittler}{ltdermittler}
%M
\DeclareMathOperator{\Mulder}{Mulder}
%N
\DeclareMathOperator{\Name}{Name}
%O
\DeclareMathOperator{\offen}{offen}
%P
\DeclareMathOperator{\Pathologie}{Pathologie}
%Q
%R
%S
\DeclareMathOperator{\Skinner}{Skinner}
\DeclareMathOperator{\Skully}{Skully}
\DeclareMathOperator{\Sonderermittler}{Sonderermittler}
%T
%U
%V
%X
%Z

\def\fspace{\rule{0.5cm}{0cm}}
\theoremstyle{definition}
\newtheorem{aufgabe1}{Aufgabe}

\usepackage{amsmath,amsthm,amssymb}
\usepackage[utf8]{inputenc}
\usepackage{wasysym}
\usepackage{stmaryrd}
\usepackage{nicefrac}
\usepackage{listings}
\usepackage{pictex}
\usepackage{color}
\usepackage{graphicx}
\usepackage{german}


%%%%%%%%%%%%%%%%%%%%%%%%%%%%%%%%%%%%%%%%%%%%%%%%%%%%%%%%%%%%%%%%%%%%%%%

\begin{document}

\title{Blatt 2}

\author{Pamela Fleischmann \& Daniel Schmidt}

\maketitle

\begin{aufgabe1}
a) $\lnot =(a1,c4)$ ist in $\mathcal{T}_1$ nicht beweisbar.\\
b) $\lnot do(a0)$ ist in $\mathcal{T}_2$ nicht beweisbar.
\end{aufgabe1}

\medskip

TBC

\bigskip

\begin{aufgabe1}
Gegeben seien zum Schema die Integritätsbedingungen\\
$f_1:(\exists X)(\exists Z)(\akte(X,\offen,Z))$\\
$f_2:(\forall X_1)(\forall X_2)(\forall Y)(\ltdermittler(X_1,Y)\land\ltdermittler(X_2,Y)\Rightarrow$\\ $=(X_1,X_2))$\\
$f_3:(\forall X)(\forall Y)(\ermittler(X,Y)\Rightarrow\lnot\ltdermittler(X,X))$\\
$f_4:(\forall X_1)(\forall Z)(\exists X_2)(\akte(X_1,\offen,Z)\Rightarrow\ermittler(X_2,Z)\land$\\ $\bearbeitet(X_2,X_1))$\\
a) Geben Sie für $f_1,\dots,f_4$ natürlichsprachliche Formulierungen an.\\
b) $(\mathcal{R},I,IB)$ ist erlaubt.\\
c) $f_1$ und $f_2$ sind Sätze der durch $\mathcal{R}$ und $I$ gegeben Theorie.
\end{aufgabe1}

\medskip

ad a) Die Formel $f_1$ ist natürlichsprachlich \glqq Für jede offene Akte existiert ein Ermittler und eine erforderliche Abteilung.\grqq . Mit
der zweiten Formel wird sichergestellt, dass jeder Ermittler nur einen leitenden Ermittler hat (gäbe es zwei, müssten diese identisch sein). $f_3$
natürlichsprachlich formuliert gibt an, dass kein Ermittler leitender Ermittler von sich selbst sein kann. Die letzte Integritätsbedingung gibt
schließlich an, dass für jede offene Akte mit dem Fall $X_1$ und der erforderlichen Stelle $Z$, ein Ermittler $X_2$ existiert, der in $Z$ arbetiet und
$X_1$ bearbeitet.

\medskip

ad b) Um zu zeigen, dass $(\mathcal{R},I,IB)$ erlaubt ist, muss gezeigt werden, dass $\models_If_i$ für $i\in[4]$ gilt. Sei $\varrho$ eine Belegung.
Dann gilt
\begin{align*}
&\models_{I,\varrho}    (\exists X)(\exists Z)(\akte(X,\offen,Z))\\
&\gdw \models_{I,\varrho}    \lnot (\forall X) (\lnot (\exists Z)(\akte(X,\offen,Z)))\\
&\gdw \not\models_{I,\varrho}(\forall X)(\lnot(\exists Z)(\akte(X,\offen,Z))).
\end{align*}
Sei $\varrho_1$ eine Belegung, die sich von $\varrho$ höchstens in der Belegung von $X$ unterscheidet. Damit folgt
\begin{align*}
&\models_{I,\varrho} f_1\\
&\gdw\not\models_{I,\varrho_1} \lnot (\exists Z)(\akte(X,\offen,Z))\\
&\gdw\not\models_{I,\varrho_1} (\forall Z)\lnot (\akte(X,\offen,Z)).
\end{align*}
Sei $\varrho_2$ eine Belegung, die sich von $\varrho_1$ höchstens in der Belegung von $Z$ unterscheidet. Damit folgt
\begin{align*}
&\models_{I,\varrho} f_1\\
&\gdw\not\models_{I,\varrho_2} \lnot (\akte(X,\offen,Z))\\
&\gdw\lnot\models_{I,\varrho_2} (||X||_{I,\varrho_2},||\offen||_{I,\varrho_2},||Z||_{I,\varrho_2})\not\in\ext(\akte)\\
&\gdw\models_{I,\varrho_2}(||X||_{I,\varrho_2},\offen,||Z||_{I,\varrho_2})\not\in\ext(\akte).
\end{align*}
Mit $\varrho_2(X)=\Einbruch$ und $\varrho_2(Z)=\Pathologie$ gilt die Behauptung.

\medskip

Weiter gilt mit der Belegung $\varrho_3$, die sich von $\varrho$ höchstens in der Belegung von $X_1$ unterscheidet
\begin{align*}
&\models_{I,\varrho} (\forall X_1)(\forall X_2)(\forall Y)(\ltdermittler(X_1,Y)\land\ltdermittler(X_2,Y)\\
&\quad \Rightarrow =(X_1,X_2))\\
&\gdw\models_{I,\varrho_3} (\forall X_2)(\forall Y)(\ltdermittler(X_1,Y)\land\ltdermittler(X_2,Y)\\
&\quad \Rightarrow =(X_1,X_2)).
\end{align*}
Sei $\varrho_4$ eine Belegung, die sich von $\varrho_3$ höchstens in der Belegung von $X_2$ unterscheidet und $\varrho_5$ eine, die sich von $\varrho_4$ höchstens
in der Belegung von $Y$ unterscheidet. Dann gilt
\begin{align*}
&\models_{I,\varrho}f_2\\
&\gdw \models_{I,\varrho_4}(\forall Y)(\ltdermittler(X_1,Y)\land\ltdermittler(X_2,Y)\\
&\quad \Rightarrow =(X_1,X_2))\\
&\gdw \models_{I,\varrho_5}(\ltdermittler(X_1,Y)\land\ltdermittler(X_2,Y)\Rightarrow =(X_1,X_2))\\
&\gdw \models_{I,\varrho_5}(\lnot \ltdermittler(X_1,Y)\lor\lnot\ltdermittler(X_2,Y)\lor =(X_1,X_2))\\
&\gdw \models_{I,\varrho_5}\lnot \ltdermittler(X_1,Y)\mbox{ oder }\lnot\models_{I,\varrho_5}\ltdermittler(X_2,Y)\\
 &\quad\mbox{ oder }\models_{I,\varrho_5}=(X_1,X_2))\\
&\gdw \not\models_{I,\varrho_5}\ltdermittler(X_1,Y)\mbox{ oder }\not\models_{I,\varrho_5}\ltdermittler(X_2,Y)\\
&\quad \mbox{ oder }\models_{I,\varrho_5}(X_1,X_2)\\
&\gdw (||X_1||_{I,\varrho_5},||Y||_{I,\varrho_5})\not\in\ext(\ltdermittler)\\
&\quad \mbox{ oder }(||X_2||_{I,\varrho_5},||Y||_{I,\varrho_5})\not\in\ext(\ltdermittler)\\
   &\quad \mbox{ oder }(||X_1||_{I,\varrho_5},||X_2||_{I,\varrho_5})\in\ext(=)\\
   &\gdw (||X_1||_{I,\varrho_5},||Y||_{I,\varrho_5})\not\in\ext(\ltdermittler)\\
   &\quad \mbox{ oder }(||X_2||_{I,\varrho_5},||Y||_{I,\varrho_5})\not\in\ext(\ltdermittler)\\
  &\quad \mbox{ oder }||X_1||_{I,\varrho_5}=||X_2||_{I,\varrho_5})
\end{align*}
Mit $\varrho_5(X_1)=\Skinner$, $\varrho_5(Y)=\Black$ ist das erste Disjunkt wahr und damit ist $I$ Modell von $f_2$.

\medskip

Sei nun $\varrho_6$ eine Belegung, die sich von $\varrho$ höchstens in der Belegung von $X$ unterscheidet und $\varrho_7$ eine Belegung, die sich höchstens
in der Belegung von $Y$ unterscheidet. Dann gilt
\begin{align*}
&\models_{I,\varrho} (\forall X)(\forall Y)(\ermittler(X,Y)\Rightarrow\lnot\ltdermittler(X,X))\\
&\gdw \models_{I,\varrho_6} (\forall Y)(\ermittler(X,Y)\Rightarrow\lnot\ltdermittler(X,X))\\
&\gdw \models_{I,\varrho_7} (\ermittler(X,Y)\Rightarrow\lnot\ltdermittler(X,X))\\
&\gdw \models_{I,\varrho_7} (\lnot \ermittler(X,Y)\lor\lnot\ltdermittler(X,X))\\
&\gdw \models_{I,\varrho_7} \lnot\ermittler(X,Y)\mbox{ oder } \models_{I,\varrho_7}\lnot\ltdermittler(X,X)\\
&\gdw \not \models_{I,\varrho_7} \ermittler(X,Y)\mbox{ oder } \not \models_{I,\varrho_7}\ltdermittler(X,X)\\
&\gdw (||X||_{I,\varrho_7},||Y||_{I,\varrho_7})\not\in\ext(\ermittler)\\
&\quad\mbox{ oder } (||X||_{I,\varrho_7},||X||_{I,\varrho_7})\not\in\ext(\ltdermittler).
\end{align*}
Mit $\varrho_7(X)=\Black$ und $\varrho_7(Y)=\Pathologie$ ist das erste Disjunkt waht und die Behauptung bewiesen.

\medskip

Sei nun $\varrho_8$ eine Belegung, die sich von $\varrho$ höchstens in der Belegung von $X_1$ unterscheidet und $\varrho_9$ eine Belegung, die sich höchstens
in der Belegung von $Z$ unterscheidet. Dann gilt
\begin{align*}
&\models_{I,\varrho} (\forall X_1)(\forall Z)(\exists X_2)(\akte(X_1,\offen,Z)\Rightarrow\ermittler(X_2,Z)\\
&\quad \land\bearbeitet(X_2,X_1)) \\
&\gdw \models_{I,\varrho_8} (\forall Z)(\exists X_2)(\akte(X_1,\offen,Z)\Rightarrow\ermittler(X_2,Z)\\
&\quad\land\bearbeitet(X_2,X_1)) \\
&\gdw \models_{I,\varrho_9} (\exists X_2)(\akte(X_1,\offen,Z)\Rightarrow\ermittler(X_2,Z)\\
&\quad\land\bearbeitet(X_2,X_1)) \\
&\gdw \not \models_{I,\varrho_9} (\forall X_2)\lnot (\akte(X_1,\offen,Z)\Rightarrow\ermittler(X_2,Z)\\
&\quad\land\bearbeitet(X_2,X_1)).
\end{align*}
Sei weiter $\varrho_{10}$ eine Belegung, die sich von $\varrho_9$ höchstens in der Belegung von $\varrho_9$ unterscheidet. Damit folgt
\begin{align*}
&\models_{I,\varrho} f_4\\
&\gdw \not \models_{I,\varrho_{10}} \lnot (\akte(X_1,\offen,Z)\Rightarrow\ermittler(X_2,Z)\\
&\quad\land\bearbeitet(X_2,X_1))\\
&\gdw \not \models_{I,\varrho_{10}} (\akte(X_1,\offen,Z)\land(\lnot \ermittler(X_2,Z)\\
&\quad\lor\lnot\bearbeitet(X_2,X_1)))\\
&\gdw \not \models_{I,\varrho_{10}} \akte(X_1,\offen,Z)\mbox{ und }\not \models_{I,\varrho_{10}}(\lnot \ermittler(X_2,Z)\\
&\quad\lor\lnot\bearbeitet(X_2,X_1)))\\
&\gdw \not \models_{I,\varrho_{10}} \akte(X_1,\offen,Z)\mbox{ und }(\not \models_{I,\varrho_{10}}\lnot \ermittler(X_2,Z)\\
&\quad\mbox{ oder }\lnot \models_{I,\varrho_{10}}\lnot\bearbeitet(X_2,X_1)))\\
&\gdw (||X_1||_{I,\varrho_{10}},||\offen||_{I,\varrho_{10}},||Z||_{I,\varrho_{10}})\not\in\ext(\akte)\\
&\quad\mbox{ und }
( (||X_2||_{I,\varrho_{10}},||Z||_{I,\varrho_{10}})\in\ext(\ermittler)\\
&\quad\mbox{ oder } (||X_2||_{I,\varrho_{10}},||X_1||_{I,\varrho_{10}})\in\ext(\bearbeitet) ).
\end{align*}
Sei $\varrho_{10}(X_2)=\Black,\varrho_{10}(Z)=\Sonderermittler$. Für jede Belegung von $X_1$ ist $f_4$ wahr.

\medskip

ad c) Mit den Axiomen aus der Vorlesung gilt
\[
\mathcal{T}\vdash \akte(\Einbruch,\offen,\Pathologie).
\]
Wird zweimal die Partikularisierungsregel (Oberschelp) angewendet, so gilt 
\[
\mathcal{T}\vdash(\exists X)(\exists Z)(\akte(X,\offen,Z)).
\]
Gehe nun davon aus, dass $\mathcal{T}\vdash\wle(X_1,Y)$ und $\mathcal{T}\vdash\wle(X_2,Y)$ gilt. Mit Oberschelp ist nun $\mathcal{T}\vdash(X_1,X_2)$ zu zeigen.
Es gilt mit den Axiomen der Vorlesung
\begin{align*}
\mathcal{T} &\vdash \wle(X_1,Y)\land(\Name(\Mulder)\lor\Name(\Skully))\\
&\quad \land =(Y,name)\land =(X_1,\Skinner)\land\wle(X_2,Y)\land =(X_2,\Skinner).
\end{align*}
Mit der Transitivität und Kommutativität von $=$ folgt $\mathcal{T}\vdash=(X_1,X_2)$. Mit der Konjunktionsregel und der Generalisierungsregel folgt
die Behauptung.

\bigskip

\begin{aufgabe1}
Formulieren Sie folgende Integritätsbedingungen im DRC:\\
a) Es muss eine Vorlesung geben, die keine Voraussetzung benötigt.\\
b) Keine Vorlesung darf sich selbst als Voraussetzung haben.\\
c) Jede Vorlesung wird nur von einem Dozenten gehalten.\\
d) In STUDENT ist Wohnsitz mehrwertig abhängig von Student.\\
e) Mindestens eine der Vorlesungen InfI, InfII und InfIII muss angeboten werden.
\end{aufgabe1}

ad a) $(\exists v)(\exists d)(\mbox{ANGEBOT}(v,d)\land(\forall v')(\forall c)(\mbox{VORAUSSETZUNG}(v',c)\Rightarrow v\neq v')$\\

\medskip

ad b) Diese Anfrage ist nicht für beliebige Zustände konkret formulierbar, da die tranistive Hülle einer binären Relation in PL1-Logik nicht ausdrückbar ist. Hierzu würde man eine Anfragesprache benötigen die Rekursion zulässt.

\medskip

ad c) $(\forall v)(\forall d_1)(\forall d_2)(\mbox{ANGEBOT}(v,d_1)\land\mbox{ANGEBOT}(v,d_2)\Rightarrow d_1=d_2)$\\

\medskip

ad d) $(\forall s)(\forall w_1,w_2)(\forall f_1,f_2)(\mbox{STUDENT}(s,w_1,f_1)\land\mbox{STUDENT}(s,w_2,f_2)\Rightarrow(\exists w)(\exists f)(\mbox{STUDENT}(s,w,f)\land w=w_2\land f=f_1)$

\medskip

ad e) $(\exists v)(\exists d)(\mbox{ANGEBOT}(v,d)\land (v=INFI\lor v=INFII\lor v=INFIII))$

\end{document}
